\begin{figure}[H]
\centering
\includegraphics[width=0.45\textwidth]{\FCNCFigures/xTFW/showFake_samesign/reg2mtau1bnjss/tau_pt_1.pdf}
\put(-100, 140){\textbf{(a)}}
\put(-120, 130){\footnotesize{SS CR}}
\includegraphics[width=0.45\textwidth]{\FCNCFigures/xTFW/showFake_samesign/reg2mtau1bnjss/subleading_index_bin.pdf}
\put(-100, 140){\textbf{(b)}}
\put(-120, 130){\footnotesize{SS CR}}\\
\includegraphics[width=0.45\textwidth]{\FCNCFigures/xTFW/showFake_sideband/reg2mtau1bnjos/tau_pt_1.pdf}
\put(-100, 140){\textbf{(c)}}
\put(-120, 130){\footnotesize{OS CR}}
\includegraphics[width=0.45\textwidth]{\FCNCFigures/xTFW/showFake_sideband/reg2mtau1bnjos/subleading_index_bin.pdf}
\put(-100, 140){\textbf{(d)}}
\put(-120, 130){\footnotesize{OS CR}}
\caption{ The distributions of sub-leading $\tau$ $\pt$ and sub-leading tau index (bin 1-6 are for 1 prong, 7-12 are for 3 prong; bin 1-3, 7-9 are for $|\eta|<1.37$ with 3 $\pt$ slices and the rest are for $|\eta|>1.52$) in the SS CR and OS CR. The fake estimation of the index is perfect by definition but in some high $\pt$ bins MC is higher than data where the estimation is treated as 0.}
\label{fig:ffsys}
\end{figure}

