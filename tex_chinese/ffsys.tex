\begin{figure}[H]
\centering
\includegraphics[width=0.48\textwidth]{\FCNCFigures/xTFW/showFake_samesign/reg2mtau1bnjss/tau_pt_1.pdf}
\put(-100, 140){\textbf{(a)}}
\put(-120, 130){\footnotesize{SS CR}}
\includegraphics[width=0.48\textwidth]{\FCNCFigures/xTFW/showFake_samesign/reg2mtau1bnjss/subleading_index_bin.pdf}
\put(-100, 140){\textbf{(b)}}
\put(-120, 130){\footnotesize{SS CR}}\\
\includegraphics[width=0.48\textwidth]{\FCNCFigures/xTFW/showFake_sideband/reg2mtau1bnjos/tau_pt_1.pdf}
\put(-100, 140){\textbf{(c)}}
\put(-120, 130){\footnotesize{OS CR}}
\includegraphics[width=0.48\textwidth]{\FCNCFigures/xTFW/showFake_sideband/reg2mtau1bnjos/subleading_index_bin.pdf}
\put(-100, 140){\textbf{(d)}}
\put(-120, 130){\footnotesize{OS CR}}
\caption{SS CR控制区(上)和OS CR控制区(下)Subleading $\tauhad$ $\pt$的分布(左)和Subleading $\tauhad$的分块(右)。$\tauhad$的分块图中bin 1-6为1 prong $\tauhad$,bin 7-12为3 prong $\tauhad$; bin 1-3和7-9为分别为$|\eta|<1.37$ 区域的个3 $\pt$切片,其余的为$|\eta|>1.52$区域。在控制区中,根据定义,分块图的数据和本底估计是完全符合的,但是由于部分低统计量的bin其即使不加入Fake tau的估计,由于统计涨落或者系统不确定度,MC就比数据还多,对于这些bin其Fake tau的本底使用保守估计,设置为0。}
\label{fig:ffsys}
\end{figure}

