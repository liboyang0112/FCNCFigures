\begin{figure}[H]
\centering
\includegraphics[width=0.48\textwidth]{\FCNCFigures/xTFW/showFake_samesign/reg2mtau1bnjss/tau_pt_1.pdf}
\put(-100, 140){\textbf{(a)}}
\put(-120, 130){\footnotesize{SS CR}}
\includegraphics[width=0.48\textwidth]{\FCNCFigures/xTFW/showFake_samesign/reg2mtau1bnjss/subleading_index_bin.pdf}
\put(-100, 140){\textbf{(b)}}
\put(-120, 130){\footnotesize{SS CR}}\\
\includegraphics[width=0.48\textwidth]{\FCNCFigures/xTFW/showFake_sideband/reg2mtau1bnjos/tau_pt_1.pdf}
\put(-100, 140){\textbf{(c)}}
\put(-120, 130){\footnotesize{OS CR}}
\includegraphics[width=0.48\textwidth]{\FCNCFigures/xTFW/showFake_sideband/reg2mtau1bnjos/subleading_index_bin.pdf}
\put(-100, 140){\textbf{(d)}}
\put(-120, 130){\footnotesize{OS CR}}
\caption{ The distributions of sub-leading $\tau$ $\pt$ and sub-leading tau index (bin 1-6 are for 1 prong, 7-12 are for 3 prong; bin 1-3, 7-9 are for $|\eta|<1.37$ with 3 $\pt$ slices and the rest are for $|\eta|>1.52$) in the SS CR(a,b) and OS CR(c,d) in hadronic channel. The fake estimation of the index is perfect but in some high $\pt$ bins MC is higher than data where the estimation is treated as 0.Only statistical uncertainties are being shown. Underflow and overflow bins are included respectively in the first and last bins. Empty data bins here are always blinded based on our strategy.}
\label{fig:ffsys}
\end{figure}

